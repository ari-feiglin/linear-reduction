\documentclass{llncs}

%\usepackage[margin=1in]{geometry}
\usepackage{amsmath, amssymb, mathtools}
\usepackage{enumitem}
\usepackage{float}
\usepackage{hyperref}
\usepackage{listings}
\usepackage{graphicx}
%\usepackage{mdframed}
%\usepackage{cleveref}

\def\labelitemi{$\top$}
\def\labelenumi{(\textbf{\arabic{enumi}})}

\usepackage{tikz}
\usetikzlibrary{trees}

\newcounter{algo}

\floatstyle{ruled}
\newfloat{algo}{h}{aux}
\floatname{algo}{Algorithm}

\input preamble

%\newcommand{\yoni}[1]{{\begin{mdframed}[linecolor=red]{\color{red}#1}\end{mdframed}}}
%\newcommand{\ari}[1]{{\begin{mdframed}[linecolor=blue]{\color{blue}#1}\end{mdframed}}}
\def\yoni#1{\par\noindent\fbox{\color{red}#1}\par}
\def\ari#1{\par\noindent\fbox{\color{blue}#1}\par}
\begin{document}

\title{Beta-Reduction}
\author{Ari Feiglin\inst{1}\and Yoni Zohar\inst{1}}
\institute{Bar-Ilan University, Ramat Gan, Israel}

\maketitle

\begin{abstract}

    In this paper we discuss an alternative approach to parsing context-free languages.
    \keywords{Context-Free Languages\and Parsing}

\end{abstract}

\section{$\beta$-Reducability}

In this section we investigate the formal language-theoretical properties of $\beta$-reduction.
In particular we discuss what sorts of languages can be $\beta$-reduced.
The definition of $\beta$-reduction in this section differs slightly from our previous definition, but the core principal
is preserved.
In particular, we will forgo priorities, which as we will see are beneficial implementation-wise, but unnecessary theoretically.

\subsection{The $\beta$-Reduction Hierarchy}

We begin by defining multiple classes of languages.
These are all special cases of the class we will begin to define now.

\begin{definition}

    Let $n>0$ be an integer, then an \emph{$n$-nondeterministic $\beta$-reduction schema} (shortened to $\nNBR$ schema) is
    a tuple $B=(\Sigma,\Gamma,\square,\top,\hb)$ where
    \begin{enumerate}
        \item $\Sigma$ is a finite alphabet of terminals;
        \item $\Gamma$ is a finite alphabet of variables (we will assume that $\Sigma\subseteq\Gamma$, or that there exists a
            clear mapping of $\Sigma$ into $\Gamma$);
        \item $\square$ is a symbol in $\Gamma$ (the string ends-marker);
        \item $\top$ is a symbol in $\Gamma$, or the empty word;
        \item $\hb$ is an \emph{initial $\beta$-reducer}, which is a function
        $$ \hb\colon\Gamma\times(\Gamma\cup\set\epsilon)\to\P(\Gamma^{\leq n}\cup\set\perp) $$
        Where $\perp$ is a symbol distinct from any other previously discussed symbol.
        Equivalently, we can view $\hb$ as a relation
        $$ \hb \subseteq \Gamma\times(\Gamma\cup\set\epsilon)\times\bigl(\Gamma^{\leq n}\cup\set\perp\bigr) $$
    \end{enumerate}

\end{definition}

Let us focus on the initial $\beta$-reducer.
It takes as input two symbols and outputs a set of strings in $\Gamma$ of length $\leq n$, and the set may also
contain the symbol $\perp$.
$\top$ is the target string, if we can reduce the string to $\top$ we consider the string part of the language.
And $\perp$ says that you can non-deterministically skip the reduction.

So for example if
$$ \hb(a,b) = \set{c,d,\perp},\quad \hb(b,a) = \set{c} $$
the string $aba$ can be reduced (in a single step) to one of the following:
$$ ca\quad da\quad ac $$
The first two result from reducing the first two characters ($a,b$) to one of their potential products, the third results from
skipping the first reduction (since $\perp\in\hb(a,b)$) and reducing the second two characters ($b,a$).

Let's formally define this process of reduction.
We assume that $\hb$ never returns an empty set (if we want to say that we can't reduce two symbols, we can return $\set\perp$).
Furthermore, we define concatenation of two sets of strings component-wise:
$A\circ B=\set{\xi_1\circ\xi_2}[\xi_1\in A,\xi_2\in B]$.
Where $\circ$ is the concatenation operator, which we will from now on leave out writing explicitly.
Furthermore, concatenation between a string and a set is done in the obvious way: $\xi A=\set\xi A$ and $A\xi=A\set\xi$.
Note that $\varnothing A=A\varnothing=\varnothing$, while $\epsilon A=A\epsilon=A$.

\begin{definition}

    Let $B$ be an $n$-\NBR, then we define its $\beta$-reduction function as follows.
    The function is $\beta\colon\Gamma^*\to\P_{<\omega}(\Gamma^*)$ and defined as follows:
    \begin{enumerate}
        \item $\beta(\sigma)=\hb(\sigma,\epsilon)-\set\perp$ if non-empty, and otherwise $\set\sigma$;
        \item For a string $\sigma_1\sigma_2\xi$, we first add
        $\bigl((\hb(\sigma_1,\sigma_2)-\set\perp)\xi\bigr)\cup\bigl((\hb(\sigma_1,\epsilon)-\set\perp)\sigma_2\xi\bigr)$ to
        the result.
        Then if $\perp\in\hb(\sigma_1,\sigma_2)\cup\hb(\sigma_1,\epsilon)$ we also add $\sigma_1\beta(\sigma_2\xi)$.
    \end{enumerate}
    This extends to a function $\beta\colon\P_{<\omega}(\Gamma^*)\to\P_{<\omega}(\Gamma^*)$
    in the natural way.
    We then define the relation $\xi\to\xi'$ to mean that $\xi'\in\beta(\xi)$; and we define $\sto$ to be the reflexive and
    transitive closure of $\to$.

    We then define the language of $B$ (or the language \emph{recognized} by $B$) to be
    $$ \L(B) = \set{w\in\Sigma^*}[\square w\square\sto\top] $$
    We define the class $\nNBR$ to be the class of all languages recognized by $\nNBR$ schemas.
    And we define $\sNBR=\bigcup_n\nNBR$.

\end{definition}

Note that an $\sNBR$-schema may be defined similar to an $\nNBR$-schema, but where the initial $\beta$-reducer returns finite
subsets of $\Gamma^*$ (denoted $\P_{<\omega}(\Gamma^*)$); or equivalently $\hb$ is a finite subset
$$ \hb \subseteq \Gamma\times(\Gamma\cup\set\epsilon)\times\Gamma^* $$
This is since $\hb$ will return a longest string of some length $n$, and then the schema is an $\nNBR$ schema.

Furthermore, we don't gain any power if we define $\L(B)$ as one of the following:
\begin{align*}
    \L(B) &= \set{w\in\Sigma^*}[\square w\sto\top],\\
    \L(B) &= \set{w\in\Sigma^*}[w\square\sto\top],\\
    \L(B) &= \set{w\in\Sigma^*}[w\sto\top]
\end{align*}

Notice that $\sNBR=\pNBR2$; this is not difficult to prove.
Take an $\nNBR$-schema $B=(\Sigma,\Gamma,\square,\top,\hb)$ and define a $\pNBR2$-schema
$B_2=(\Sigma,\Gamma_2,\square,\top,\hb_2)$ as follows: $\Gamma_2$ will contain $\Gamma$ as well as every prefix of strings
returned by $\hb$ (each prefix is distinct, if two strings have the same prefix, or the same string is returned by two different
reductions, we consider them distinct).
Given such a prefix $\xi$, denote the symbol in $\Gamma_2$ by $\mbox\xi$.
Then if $\sigma_1\cdots\sigma_n\in\hb(\tau_1,\tau_2)$, we define
$$ \hb_2(\tau_1,\tau_2) = \mbox{\sigma_1\cdots\sigma_n},\quad
\hb_2(\mbox{\sigma_1\cdots\sigma_k},\epsilon)=\set{\mbox{\sigma_1\cdots\sigma_{k-1}}\sigma_k}\hbox{ for $k\leq n$} $$
If $\perp\hb(\tau_1,\tau_2)$, then $\perp\in\hb_2(\tau_1,\tau_2)$ as well.

\begin{lemma}

    $\sNBR=\pNBR2$

\end{lemma}

\subsection{Context-Free Languages and $\pNBR1$}

\begin{definition}

    A \emph{context-free grammar} is a tuple $G=(\Sigma,V,S,P)$ where
    \begin{enumerate}
        \item $\Sigma$ is a finite alphabet of terminals;
        \item $V$ is a finite alphabet of variables;
        \item $S\in V$ is the initial variable;
        \item $P\subseteq V\times(\Sigma\cup V)^*$ is a set of production rules, where we write $A\vdash_G\alpha$ in place of
        $(A,\alpha)\in P$ (and we may omit the subscript from $\vdash$ when $G$ is understood).
    \end{enumerate}
    We extend the definition of $\vdash$ to a relation $(\Sigma\cup V)^*\times(\Sigma\cup V)^*$, where if
    $\alpha A\beta$ is a string of terminals and variables, $A\in V$ such that $A\vdash\gamma$, then
    $\alpha A\beta\vdash\alpha\gamma\beta$.
    $\vdash^*$ is the reflexive and transitive closure of $\vdash$.
    The language of $G$ is defined to be
    $$ \L(G) = \set{w\in\Sigma^*}[S\vdash^*w] $$
    A language is \emph{context-free} if it is the language of some context-free grammar.
    The class of context-free languages is denoted $\CFL$.

\end{definition}

\begin{theorem}

    $\CFL\subseteq\pNBR1$

\end{theorem}

\begin{proof}

    Let $G$ be a context-free grammar in Chomsky normal form.
    We define $B=(\Sigma,\Gamma,\square,\epsilon,\hb)$, where $\Gamma=\Sigma\cup V\cup\set\square$ and for every rule
    $A\vdash_GXY$, we add the reduction $X,Y\to_BA$.
    And for every rule $A\vdash_G\sigma$, we add the reduction $\sigma,\epsilon\to_BA$.
    Then for every two symbols $X,Y$ we add the reduction $X,Y\to_B\vdash$.
    If $S\vdash_G\epsilon$ is a rule, then we also add the reduction $\square,\epsilon\to_B\epsilon$.
    We also add the rule $\square,S\to_B\epsilon$.

    Now, we claim that for $\alpha,\gamma\in\Gamma^*$, $\alpha\vdash\gamma$ if and only if $\gamma\to\alpha$.
    Indeed, if $\alpha=\alpha_1A\alpha_2$ and we use the rule $A\to XY$, then $\gamma=\alpha_1XY\alpha_2$.
    And we can use the reduction $X,Y\to A$ to get $\gamma\to\alpha_1A\alpha_2=\alpha$ (since every reduction contains $\perp$,
    we can arbitrarily skip strings).
    And if $\alpha=\alpha_1A\alpha_2$ and we use the rule $A\to\sigma$, then $\gamma=\alpha_1\sigma\alpha_2$ and of course we
    have $\gamma\to\alpha$.

    Proving the other direction is similar.
    Thus we have $\alpha\vdash\gamma$ if and only if $\gamma\to\alpha$.
    Thus $\alpha\vdash^*\gamma$ if and only if $\gamma\to^*\alpha$.
    In particular $S\vdash^*\gamma$ (i.e. $\gamma\in\L(G)$) if and only if $\gamma\to^*S$.
    This means that $\square\gamma\to^*\square S\to\epsilon$, so $\gamma\in\L(B)$.
    If $\square$ is reduced before $\gamma$ gets to $S$, we will have the case $\square\gamma\to^*S$, which won't reduce.
    So in either case we have that $S\vdash^*\gamma\iff\square\gamma\to^*S$.
    This works for $\gamma=\epsilon$ as well.
    So we have $\L(G)=\L(B)$ as required.
    \qed

\end{proof}

\subsection{Rewrite Systems}

We first begin by taking a different perspective on $\beta$-reduction.

\begin{definition}

    An \emph{$n$-$\beta$-rewrite system} ($\BRW n$-system) is a tuple $B=(\Sigma,V,\hb)$, where
    \begin{enumerate}
        \item $\Sigma$ is a finite alphabet;
        \item $V$ is a finite set of variables, disjoint from $\Sigma$;
        \item $\hb$ is the \emph{initial $\beta$-function} a function
        $$ \hb\colon \Gamma\times(\Gamma\cup\set\epsilon)\to\P(\Gamma^{\leq n}) $$
        Where $\Gamma=\Sigma\cup V$.
        Equivalently, a $\hb$ is a relation
        $$ \hb \subseteq \Gamma\times(\Gamma\cup\set\epsilon)\times\Gamma^{\leq n} $$
    \end{enumerate}
    We can also call an $\BRW n$-system a $\sBRW$-system.

\end{definition}

\begin{definition}

    Let $B$ be a $\BRW n$-system, then we define the binary relation $\to_B$ on $\Gamma^*$ as follows: we write
    $\alpha\to_B\gamma$ if $\alpha=\alpha_1\sigma\tau\alpha_2$, where $\sigma\in\Gamma$ and $\tau\in\Gamma\cup\set\epsilon$ and
    $\gamma=\alpha_1\hb(\sigma\tau)\alpha_2$.
    We then define $\to_B^*$ to be the transitive and reflexive closure of $\to$.

    Further, we define $\tto_B$ and $\tto^*_B$ to be restrictions of $\to_B$ and $\to^*_B$ to $\Sigma^*$.

    Call two $\sBRW$s $B_1$ and $B_2$ \emph{equivalent} if ${\tto_{B_1}^*}={\tto_{B_2}^*}$

\end{definition}

Notice that unlike with $\sNBR$-schemas, in $\sBRW$-systems all reductions may be done non-deterministically.
E.g. given a string $abc$, we can reduce either $a,b$ or $b,c$.
As we will see, this doesn't impose any practical limitations for $n\geq2$.

Using a proof similar to the one for $\sNBR$-schemas, we get the following result:

\begin{lemma}

    Every $\sBRW$-system is equivalent to a $\BRW2$-system.

\end{lemma}

\begin{definition}

    A \emph{semi-Thue system} (or a \emph{string rewriting system}) is a pair $T=(\Sigma,R)$ where $\Sigma$ is an alphabet
    (not necessarily finite), and $R$ is a binary relation on $\Sigma^*$.
    Instead of $(u,v)\in R$ we write $u\vdash_Tv$.
    We extend this relation to other $\Sigma$-strings as follows: if $u\vdash_Tv$ then for any $\alpha,\gamma\in\Sigma^*$ we
    declare $\alpha u\gamma\vdash_T\alpha v\gamma$.
    We let $\vdash_T^*$ be the reflexive and transitive closure of $\vdash_T$.

    If both $\Sigma$ and $R$ are finite, we call $T$ \emph{finite}.

\end{definition}

Finite semi-Thue systems are Turing complete.
This is clear since every formal grammar forms a finite semi-Thue system.

\begin{lemma}

    $\sBRW$ (and thus $\BRW2$) is Turing complete.

\end{lemma}

\begin{proof}

    It is sufficient to show that for every finite semi-Thue system $T=(\Sigma,R)$, there exists a $\sBRW$-system $B$ such that
    ${\tto_B^*}={\vdash_T^*}$.

    Let $B=(\Sigma,V,\hb)$, where for every string $u$ in $R$ we add all the prefixes of $u$ to $V$.
    We can assume that for each $u$ of $R$ the substrings added are unique (and we view multiple occurrences of $u$ in $R$ as
    distinct).
    We write elements of $V$ in boxes to distinguish them.

    Now for $u\vdash_Tv$, let us suppose that $u\neq\epsilon$.
    So let $u=\sigma_1\cdots\sigma_n$ and $v=\tau_1\cdots\tau_m$, then we add rules
    $$
    \sigma_1,\epsilon\to_B\mbox{\sigma_1},\qquad
    \mbox{\sigma_1\cdots\sigma_k},\sigma_{k+1}\to_B\mbox{\sigma_1\cdots\sigma_{k+1}}
    $$
    and
    $$
    \mbox{\sigma_1\cdots\sigma_n},\epsilon\to_B\tau_1\cdots\tau_m
    $$

    For rules of the form $\epsilon\vdash_Tv$, we add rules $\sigma,\epsilon\to\sigma v$ for every $\sigma\in\Sigma$.

    Clearly if $u\vdash_Tv$ then $u\tto_B^*v$, and vice versa (for the converse we of course have $u\vdash_T^*v$).
    Thus we have the desired result.
    \qed

\end{proof}

\begin{theorem}

    Every recursively enumerable language is $\pNBR2$, and conversely every $\sNBR$ language is recursively enumerable.

\end{theorem}

If $T$ is a semi-Thue system, it has an \emph{inverse}: a semi-Thue system $T^r$ where
$$ u\vdash_{T^r}v \iff v\vdash_Tu $$

\begin{proof}

    Given a recursively enumerable language $L$, there exists some finite semi-Thue system which generates it (meaning
    $u\in L$ iff $S\vdash_T^*u$ for some $S\in\Sigma$).
    Then if we look at the inverse semi-Thue system, $u\in L$ iff $u\vdash_{T^r}^*S$.
    Let $B$ be the $\BRW2$-system equivalent to $T^r$.
    Then we make it into a $\pNBR2$-schema by setting $\Gamma=\Sigma\cup V$, and $\top=S$.
    Then we have that
    $$ u\in L \iff u\vdash_{T^r}^*S \iff u\tto_B^*\top \iff u\in\L(B) $$
    as required.

    The converse is obvious: every $\sNBR$-schema can obviously be simulated by a nondeterministic Turing machine.
    \qed

\end{proof}

So we see that for $n\geq2$, whether or not $\BRW2$-systems can always or can only sometimes skip a reduction is immaterial:
both are Turing complete.
This justifies us not adding $\bot$ to $\sBRW$-systems.

\subsection{Deterministic $\beta$-Reduction}

\begin{definition}

    An $\nNBR$-schema $B$ is \emph{deterministic} if for every $\sigma\in\Gamma,\tau\in\Gamma\cup\set\epsilon$, we have that
    $\lvert\hb(\sigma,\tau)\rvert=1$.
    That is $\hb(\sigma,\tau)=\set\alpha$ or $\hb(\sigma,\tau)=\set\perp$.
    Furthermore, we require that if $\hb(\sigma,\tau)$ is defined (not equal to $\set\perp$) for some $\tau$,
    $\hb(\sigma,\epsilon)$ is undefined.

    We call a deterministic $\nNBR$-schema an $\nBR$-schema.
    Equivalently an $\nNBR$-schema is one where $\hb$ is a partial function
    $$ \hb\colon\Gamma\times(\Gamma\cup\set\epsilon) \partial \Gamma^{\leq n} $$
    (Where an image not being defined is equivalent to defining it to be $\set\perp$.)
    such that if $\hb(\sigma,\tau)$ is defined, $\hb(\sigma,\epsilon)$ is not.

    The class of languages accepted by $\nBR$-schemas is denoted $\nBR$.
    Let $\sBR=\bigcup_{n>0}\nBR$.

\end{definition}

Our goal in this section is to understand the relationship between $\nBR$ and the class of deterministic context-free languages.
We denote this class by $\DCFL$.
Recall that a deterministic context-free language is one that can be accepted by a deterministic pushdown automaton.
Our previous proof techniques do not work for the deterministic case, as they rely on the ability to non-deterministically
choose whether or not to reduce.

Though notice that our proof that $\sNBR=\pNBR2$ converts deterministic schemas to deterministic schemas.
So we immediately have that

\begin{lemma}

    $\sBR=\pBR2$

\end{lemma}

We will use an equivalent definition of deterministic context-free languages to show that $\DCFL\subseteq\pBR2$.

\begin{definition}

    An \emph{LR parser} is a tuple $P=(\Sigma,Q,V,\action,\goto,q_0,\square)$ where
    \begin{enumerate}
        \item $\Sigma$ is a finite alphabet of terminals;
        \item $Q$ is a finite set of states (disjoint from $\Sigma$);
        \item $V$ is a finite set of variables (disjoint from $\Sigma$ and $Q$);
        \item $\action$ is a partial function
        \begin{multline*}
            \action\colon Q\times\Sigma\cup\set\square\partial\\
            \set{\shift(q)}[q\in Q]\cup\set{\reduce(A,n)}[A\in V,n\in{\bb Z}_{\geq0}]\cup\set\accept
        \end{multline*}
        ($\shift,\reduce,\accept$ are simply distinct symbols.)
        \item $\goto$ is a partial function
        $$ \goto\colon Q\times V \partial Q $$
        \item $q_0\in Q$ is the initial state;
        \item $\square$ is a symbol not in $\Sigma\cup Q\cup V$, the string endmarker.
    \end{enumerate}

\end{definition}

\begin{definition}

    A \emph{configuration} of an LR parser $P$ is a pair $c\in Q^*\times(\Sigma\cup\set\square)^*$, i.e. a pair
    of the form $(p_0\cdots p_n\;;\;\sigma_1\cdots\sigma_m)$.
    We define $P(c)$ as follows: suppose $c=(q_0\cdots q_m\;;\;\sigma_1\cdots\sigma_n\square)$, then
    \begin{enumerate}
        \item if $\action(q_m,\sigma_1)=\shift(p)$ then
        $$ P(c) = (q_0\cdots q_mp\;;\;\sigma_2\cdots\sigma_n\square) $$
        \item if $\action(q_m,\sigma_1)=\reduce(A,k)$ then let $p=\goto(q_{m-k},A)$ and
        $$ P(c) = (q_0\cdots q_{m-k}p\;;\;\sigma_1\cdots\sigma_n\square) $$
        if $\goto(p_{m-k},A)$ is undefined, $P(c)=\bot$ (a special configuration invariant under $P$);
        \item if $\action(q_m,\sigma_1)=\accept$ then $P(c)=\top$ (also invariant under $P$);
        \item if $\action(q_m,\sigma_1)$ is undefined then $P(c)=\bot$.
    \end{enumerate}
    The language of $P$ is
    $$ \L(P) = \set{w\in\Sigma^*}[\hbox{there exists an $n>0$ such that $P^n(q_0\;;\;w\square)=\top$}] $$

\end{definition}

A well-known theorem is the following:

\begin{theorem}

    A language $\L$ is deterministic context-free if and only if there exists an LR parser such that $\L(P)=\L$.

\end{theorem}

\begin{lemma}

    $\DCFL\subseteq\sBR$

\end{lemma}

\begin{proof}

    Let $P$ be an LR parser.
    We define a $\sBR$-schema $(\Sigma,\Gamma,q_0,\top,\hb)$ where $\Gamma$ consists of $\Sigma\cup Q$ and for every
    $\reduce(A,n)$ in the image of $\action$, $M_{k,A}$ for $k\leq n$ and $A\in V$, and we add $\perp,\top$ as well.

    For every $q\in Q$ and $\sigma\in\Sigma$,
    \begin{enumerate}
        \item if $\action(q,\sigma)=\shift(p)$ then $\hb(q,\sigma)=qp$;
        \item if $\action(q,\sigma)=\reduce(A,n)$, then $\hb(q,\sigma)=qM_{n,A}\sigma$;
        \item if $\action(q,\sigma)=\accept$, then $\hb(q,\sigma)=\top$;
        \item if $\action(q,\sigma)$ is undefined, then $\hb(q,\sigma)=\bot$;
        \item $\hb(q,M_{n,A})=M_{n-1,A}$ if $n>0$;
        \item $\hb(q,M_{0,A})=q\goto(q,A)$ if defined, otherwise $\perp$;
        \item $\hb(\sigma,\top)=\hb(\top,\sigma)=\top$.
    \end{enumerate}
    We also add $\hb(q,\square)$ according to $\action(q,\square)$.
    We can assume that this is not a $\shift$ operation.

    It's not hard to see that for a given configuration $c=(v;w)$, we have that $P(c)=(v',w')$ then $vw\to^* v'w'$.
    And if $vw\to^*v'w'$ then $P^k(c)=(v',w')$ for some $k\geq0$.
    Thus we get $\L(P)=\L(B)$ as required.
    \qed

\end{proof}

Since $\sBR=\pBR2$, we get

\begin{lemma}

    $\DCFL\subseteq\pBR2$

\end{lemma}

Notice here that the $\sBR$-schema defined in the lemma above is a $\pBR3$-schema.
But under the assumption that there are no actions of the form $\reduce(A,0)$ (which corresponds to a rule
$A\to\epsilon$ in a grammar), we can alter it to be a $\pBR2$-schema.
We replace rule $2$ with $\hb(q,\sigma)=M_{n-1,A}\sigma$.
Now, we define the following variation of $\pBR2$-schemas:

\begin{definition}

    A $\pBR2$-schema $B$ is \emph{non-increasing} if for all $\sigma\in\Gamma$,
    $\abs{\hb(\sigma,\epsilon)}\leq1$ if defined.
    That is, the string being reduced can never increase in length.

\end{definition}

\begin{definition}

    Let $B$ be a non-increasing $\pBR2$-schema.
    It is \emph{decreasing} if the following two conditions hold:
    \begin{enumerate}
        \item for all $\sigma\in\Gamma$, $\beta^n(\sigma)$ never equals $\sigma$ for all $n\geq1$;
        \item for all $\sigma,\tau\in\Gamma$, $\beta^n(\sigma\tau)$ never begins with $\sigma$.
    \end{enumerate}

\end{definition}

It is immediate from this definition that a decreasing $\pBR2$-schema will always terminate on every input
string.

\begin{lemma}

    Every non-increasing $\pBR2$-schema has an equivalent decreasing $\pBR2$-schema.

\end{lemma}

\begin{proof}

    Let $B$ be a non-increasing $\pBR2$-schema, we construct an equivalent decreasing $\pBR2$-schema $B'$.
    Our first step we do iteratively, starting with a $\pBR2$-schema $B_1$, until $B_{\abs\Gamma}$.
    First we add a new symbol $\perp$ to $\Gamma_1$, it has no reduction rules.

    Suppose we have constructed $B_i$ ($0\leq i<\abs\Gamma$, $B_0=B$).
    For each $\sigma\in\Gamma$ define
    $$ n_\sigma = \min\set{n\geq1}[\beta^n(\sigma)=\sigma] $$
    If all $n_\sigma$s are equal to $\infty$, then set $B_{i+1}=B_i$.
    Otherwise let $\sigma$ has the minimal $n_\sigma$, which means that in the reduction
    $$ \sigma = \sigma_0 \to \sigma_1 \to \cdots \to \sigma_n = \sigma $$
    all the $\sigma_i$s are distinct, except for $i=0,n$.

    Now, add new symbols $\sigma'_1,\dots,\sigma'_{n-1}$ to $\Gamma_{i+1}$, with the reduction rules
    \begin{align*}
            \hb_{i+1}(\sigma'_j,\epsilon)&=\sigma'_{j+1},\\
            \hb_{i+1}(\tau,\sigma'_j) &= \hb(\tau,\sigma_j),\\
            \hb_{i+1}(\sigma'_j,\tau) &= \hb(\sigma_j,\tau)
    \end{align*}
    for $0\leq j\leq n-2$ ($\sigma'_0=\sigma$), as well as the rule $\hb_{i+1}(\sigma'_{n-1},\epsilon)=\bot$.
    We have thus created $B_{i+1}$.

    Now, $B^0=B_{\abs\Gamma}$ has the first property of a non-increasing schema.
    That is, if $\beta^n(\sigma)$ never equals $\sigma$.
    Furthermore, $B^0$ and $B_0=B$ are equivalent.

    We now construct $B'$ from $B^0$.
    For every reduction of the form
    $$ \sigma_0\tau_0 \to \sigma_1\tau_1 \to \cdots \to \sigma_n\tau_n = \sigma_0\tau_n $$
    we add characters $\sigma'_1,\dots,\sigma'_{n-1}$.
    We add the reduction rules
    \begin{align*}
            \hb'(\sigma'_i,\tau_i) &= \sigma'_{i+1}\tau_{i+1},\\
            \hb'(\tau,\sigma'_i) &= \hb(\tau,\sigma_i),\\
            \hb'(\sigma'_{n-1},\tau_{n-1}) &= \bot
    \end{align*}
    This is clearly decreasing, and equivalent to $B^0$.
    \qed

\end{proof}

\begin{lemma}

    Every decreasing $\pBR2$ schema takes $O(\abs\Gamma\cdot n^2)$ reductions until it terminates.

\end{lemma}

\begin{proof}

    We claim that it  must take at most $\abs\Gamma\cdot n$ reductions to reduce the string by one symbol.

    Let us call reductions of the form $\sigma_1,\sigma_2\to\tau_1\tau_2$ $2$-reductions and $\sigma\to\tau$
    $1$-reductions.

    Let us create a counter $n(i)$ for the number of $1$- and $2$- reductions at each index.
    Successive $1$- or $2$-reductions of an index can never place a symbol at that index that has already been
    there, since the schema is decreasing.
    This means that $n(i)\leq\abs\Gamma$.
    And the number of reductions in total is just $\sum_in(i)\leq n\abs\Gamma$, as required.

    Then it takes at most $n$ reductions of a symbol to terminate.
    Which means that it takes at most $O(\abs\Gamma\cdot n^2)$ reductions.
    \qed

\end{proof}

\begin{lemma}

    There exists an algorithm which executes any given decreasing $\pBR2$-schema in $O(n^2)$ time.
    This algorithm is also optimal.

\end{lemma}

\begin{proof}

    Since an execution of a decreasing $\pBR2$-schema requires $O(n^2)$ reductions, any algorithm which
    executes a $\pBR2$-schema takes $\Omega(n^2)$ time.
    Thus if we have an algorithm which runs in $O(n^2)$ time, it must be optimal.

    Our algorithm is as follows:

    \def\flag{\mathit{flag}}
    \algorithm
        \Function{Execute}{$B,w$}
            \State $\flag\gets1$
            \State $i\gets0$
            \While{$\flag$}
                \lIf{$\hb(w[i],w[i+1])=\sigma_1\sigma_2$}
                    Replace $w[i],w[i+1]$ with $\sigma_1,\sigma_2$
                \lElseIf{$\hb(w[i],w[i+1])=\sigma$}
                    Replace $w[i],w[i+1]$ with $\sigma$
                \lElseIf{$\hb(w[i],\epsilon)=\sigma$}
                    Replace $w[i]$ with $\sigma$
                \If{any of the above conditions were true, and $\sigma\neq\epsilon$}
                    \lIf{$\sigma\neq\epsilon$} $i\gets i-1$
                \Else
                    \State $i\gets i+1$
                    \lIf{$i=\abs w$} $\flag\gets0$
                \EndIf
            \EndWhile
            \State\Return $w$
        \EndFunc
    \ealgorithm

\end{proof}

Unfortunately, notice that this is not enough to show that $\DCFL\subseteq\pBR1$.
This is since we conjecture that $\pBR1\subseteq\DCFL$ yet decreasing $\pBR2$-schemas can still recognize non-$\DCFL$ languages.
In fact, they can recognize non-$\CFL$ languages!
For example, $\set{a^nb^nc^n}[n\geq0]$ is not $\CFL$ yet is $\pBR2$.

\begin{proposition}

    $\set{a^nb^nc^n}[n\geq0]$ is $\pBR2$.

\end{proposition}

\begin{proof}

    Define $\Gamma=\set{a,b,c,X,\square}$, and $\hb$ as follows:
    \begin{enumerate}
        \item $\hb(a,b)=X$: this matches the first $a$ and $b$;
        \item $\hb(a,X)=Xa$; this moves the $X$ out of the way (eventually the the leftmost side of the string) to allow
            rule $1$ to repeat;
        \item $\hb(X,c)=\epsilon$: this matches $X$ with a $c$;
        \item $\hb(\square,\square)=\top$.
    \qed
    \end{enumerate}

\end{proof}

Using a similar proof we can show that $\set{\sigma_1^n\cdots\sigma_m^n}[n\geq0]$ is $\pBR2$ for
$\Gamma=\set{\sigma_1,\cdots,\sigma_m}$.

\subsubsection{Priorities}

Now, one may be inclined to ask why we don't add priorities to this $\beta$-reduction.
It turns out that priorities, under certain assumptions, don't add any power.
Indeed, suppose we added priorities to our $\beta$-reducer, so now our initial $\beta$-reducer is of the form
$$ \hb\colon\Gamma\times(\Gamma\cup\set\epsilon)\to\P_{<\omega}((\Gamma^{\leq n}\times(\bbZ^2\to\bbZ^n))\cup\set\bot) $$
and we have an initial priority function $\pi\colon\Sigma\to\bbZ$.
Let $F$ be the set of priority functions returned by $\hb$, that is
$$ F = \set{f\colon\bbZ\times\bbZ\to\bbZ^n}[(\xi,f)\in\hb(\sigma,\tau)\hbox{ for some $\sigma,\tau$}] $$
and let $N=\mathrm{Im}\pi$ be the set of all initial priorities.
Now let us define $N_0=N$ and $N_{n+1} = N_n\cup\set{k}[n\geq m\in N_n, f\in F,k\in f(n,m)]$.
We define $N_\infty=\bigcup_{n\geq0}N_n$.
If $N_\infty$ is finite, then there exists an equivalent \BR-schema (one without priorities).

The proof of this is simple: define $B=(\Sigma,\Gamma',\square,\blacksquare,\bullet,s',\hb')$ where
\begin{enumerate}
    \item $\Gamma'=\Gamma\times N_\infty$ which is still finite.
        Elements of $\Gamma'$ will be written as $\sigma_n$ instead of $(\sigma,n)$.
    \item $s'$ maps $\sigma$ to $s(\sigma)_{\pi(\sigma)}$.
    \item If $\hb(\sigma,\tau)=(\alpha,f)$ then for $n\geq m\in N_\infty$ we define $\hb'(\sigma_n,\tau_m)=\xi_{f(n,m)}$.
        Where by $\xi_{f(n,m)}$ we mean that if $f(n,m)=(n_1,\dots,n_k)$ and $\xi=\sigma^1\cdots\sigma^k$ then
        $\xi_{f(n,m)}=\sigma^1_{n_1}\cdots\sigma^k_{n_k}$.
        Note that by definition $n_i\in N_\infty$ so this is well-defined.
\end{enumerate}
It is not hard to see that the $\beta$-reducer and this \BR-schema both generate the same language.

The requirement that $N_\infty$ be finite is a strong condition (for example, $F$ cannot include a function of the form $(n,m)\mapsto n+m$ if $N\neq\set0$).
But in practice this is not a restrictive condition, for example ILang satisfies this condition: if $F$ consists of functions of the form $k_c$ (the constant map), $\snd,\fst$, then $F_\infty$ will trivially be finite.

\end{document}

